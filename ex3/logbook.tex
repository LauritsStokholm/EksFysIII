\documentclass[a4paper, oneside, onecolumn, 11pt]{memoir}
\newenvironment{code}{\captionsetup{type=listing}}{}

%\begin{code}
%    \caption{Den skrevne Pythonkode.
%    \label{kode}}

%    \inputminted[firstline=20, lastline=30,
%        frame=single, framesep=2mm, fontsize=\footnotesize, linenos% Spanning over more than one page!
%    ]{python}{kode.py}
%\end{code}

% Preamble
\usepackage{preamble}

\graphicspath{{./Graphics/}}

\title{Logbook}
\author[1]{Anne Kirstine Knudsen\thanks{anne839i97@gmail.com}}
\author[1]{Laurits N. Stokholm\thanks{laurits.stokholm@post.au.dk}}
\affil[1]{Department of Physics and Astronomy, Aarhus University}
\renewcommand{\Affilfont}{\itshape}
\date{\today}

\begin{document}
\maketitle
%\saythanks
\section{Overview}
On this page, the reader will find the whole experiment summed up on a single
page. It should be used to get an overview before the reader moves on to the
rest of the lab script. It does not tell the reader why you need to do these
procedures, but this can be found in the following pages. It is suggested that
you have this page at hand, while you are doing the experiment.

\section{Calibration}
The calibration relates the channelnumbers, which is the output of the detector program, to measured photon energy.
\begin{enumerate}
\item Measure the activity of the calibration sources with a Geiger-counter.
\item Choose fitting calibration sources, measure these with the NaI detector.
    Measure the Cs-137 radiation source with the BGO detector. 30-60 minutes
    should be enough for these measurements.
\item Identify and read the channel number of the calibration sources
    photopeaks of the calibration sources, for both the BGO detector and the
    NaI detector.
\item Find the relation between channel number and photon energy, in each
    detector.
\end{enumerate}

\section{Coincidence measurement}
The measurement of coincidence between the two detectors is where the effects of Compton-scattering can be seen.

\begin{enumerate}
\item Set up the program for coincidence measurements. Choose an angle to
    measure at.
\item Let the measurement run for preferably more than 10 hours.  
\item Use the calibration values to merge the datafiles from the two detectors
    with ROOT-script TTree- Builder.c.
\item Use the given scripts to plot the cloudplot for the measurement.
\item Identify the area of the cloudplot where Compton-scattering has happened
    in the BGO-detector, and the Photoelectric effect has happened in the
    NaI-detector. Read the scattered-photon energy. 
\item repeat steps 1 through 5 for differing angles.
\item Compare the measured shift in energy with the theoretical value given by the Compton effect.  Results
\end{enumerate}

\section{Results}
\begin{enumerate}
\item Plot the coincidence measurements as cloudplots, and identify the overlap between Compton- scattering in the first detector (BGO), and photoeletric effect in the second detector (NaI).
\item Interpret the movement of this area at varying angles.
\end{enumerate}

\section{Experimental Procedure}
\subsection{Before you begin}
You should find the characteristic energy diagrams for those atoms you have
chosen to investigate. It is the $\gamma$–peaks of theese atoms you later will
have to compare with. When we did the experiment, we found the energy diagrams
for $Cs-137$ (1 peak), $Na-22$ (2 peaks) and $Co-60$ (2 peaks).  All
information can be found at \url{http://www.nndc.bnl.gov/}.

\subsection{Calibration}
The scintillators do not themselves measure the energy of the photons, but the power supply converts the signals to a channel number, which corresponds to a discrete energy. The purpose of the calibration is to convert the given channelnumbers from the AC–converter to a specific energy.  To do so, it is necesary to compare the data of measurements, with the known energies at a given photopeak. You can find an illustrated review of the following description in the appendix.  

\subsubsection{Hardware}
\begin{enumerate}
\item Check that both detectors are linked to the power supply, and make sure
    to note which channel each of their cabels are connected to. The power
    supply should also be connected to the computer.
\item Extend the arm between the BGO and the NaI in an angle relative to the
    BGO.
\end{enumerate}

\subsubsection{Software}
\begin{enumerate}
\item To set up the software for the calibration, open MC2Analyser.
\item Go to “Acquisition Setup”, make sure the Online Spectrum is ticked, click
    “New Board Connection”.
\item “Device Conection” will open in a new window. Click “connect”.
\item Now turn “PWR” on.
\item Go to “Acquisition Setup”, click “Coincidences”, and make sure that all
    spaces are unticked and the drop downs are all on “NONE”. Press “Apply”. Go
    to output, and fill in a directory and the name of the filename to save
    data.
\item At last, click “RUN”. Let it run for as long as possible, a couple of
    hours or more.
\end{enumerate}

The calibration has to be done for both the BGO and the NaI–detector. For the
calibration of the BGO–detector, use the $Cs-137$ source encaptured within the
lead enclosement. DO NOT start fiddling with the lead enclosement! For the
calibration of the $NaI$–detector, place a radioactive source on top of it. The
$\gamma$-radiation source can be varied. Ask your lab-instructor for the other
sources. It is recommended to use a Geiger–counter to find the most active
radiation sources. From our experience, the $Co-60$ was great as opposed to the
$Na-22$, which could barely show any photopeak with the background effects of a
$Cs-137$ radiation source.  Let the calibration run for at least 30 minutes.
Nonetheless, the longer it runs, the more precise. When the calibration is
done, click “Stop” and remember to reset curve before continuing with the rest
of the experiment. Now it is time to open the “Histogram” script in MatLab –
you will find the instructions for the script written in MatLab. To convert the
arbitary unit to a unit of energy, it is necesary to calculate the linear
proportionality constant, $\kappa$.
\begin{equation}
    E_{\text{source}} = \kappa E_{\text{Channel}}
\label{eq:unitconversion}
\end{equation}
The conversion factor should not be material dependent, but this might be a
point of interest to investigate further. To find $\kappa$, you should first
find the $\gamma$–peaks for each radiation source and then use then use the
matlabcaliscript. When calculated, the $\kappa$–factor is constant for all
measurements.

\subsubsection{Practical/ Technical notes}
One should be aware of the following:

 \renewcommand{\labelenumi}{\Roman{enumi}}
\begin{enumerate}
    \item The optical breadboard with all its components is very heavy, so take care when taking it out of the cabinet and back again.
    \item Remember laser light can be harmful, so be careful also with parasitical beams!
    \item Remember not to touch any optics on the surfaces on which light in impinging!  
    \item Do only apply voltages in the range 0-10V to the control input of the piezodriver,
        and do not drive it at a frequency of more than 200 Hz. Hence, check and 
        adjust the output of the function generator with the Pico Scope before  
        connecting it to the piezo-driver. The voltage delivered to the piezo should be a   
        factor of 10 higher than the control voltage (check backside of the piezo-driver).
    \item Remember to take clear pictures of you various setups, including the 
        electronic wiring. 
    \item At the end of each experimental session, remember to safely fix all the optical 
        elements to the breadboard in positions similar to those on Fig. 1., and bring   
        everything back in good order in the cabinets!

\end{enumerate}

\end{document}
