\section{Theory}
\subsection{Compton Scattering} Compton scattering occurs when a beam of
lightparticles, known as photons, collides with a solid target, in such a way
the photon collides with a bounded electron. The scattering process can be seen
on \cref{fig: comptonscattering}.  If the photon has an energy higher than the
binding energy of the electron, both the electron and the photon scatters. Then
the difference between the energy of the incomming and the scattered photon is
described by the Compton shift;
\begin{equation}
    \lambda - \lambda_0 = \frac{h}{m_ec}(1-\cos(\theta))
    \label{eq: comptonshift}
\end{equation}
Where $\lambda$ is the wavelength of the incoming photon, $\lambda_0$ is the
wavelength of the scattered photon, $h$ is Plancks constant, $m_e$ is the mass
of an electron, $c$ is the speed of light and $\theta$ is the angle in which
the photon is scattered. The derivation can be found in the \fxnote{appendix}.
Another effect closely related to this experiment is the photoelectric effect.
In contrast to the Compton effect, the photon energy is fully absorbed and goes
to two things; freeing the electron from its bounded state, and accelerating
the electron. The freed electron is called a photoelectron. This interaction
can of course not happen, if the photon do not have enough energy to free the
electron.
