\section{Conclusion}
The classical Rutherford scattering of hydrogenic ions, $H^+$ and ${H_2}^+$ on
a double layered target of $\SI{20}{\angstrom}$ gold and $\SI{200}{\angstrom}$
carbon have been examined.
For this a Van-de-Graaff accelerator was used to obtain energies of
$\SI{350}{\kilo\electronvolt}$. As the energy of the incomming particles are
relatively low, the Sommerfeld criterion for the classical regime is well fit.
The values are $\eta_|S| \sim 300$ for gold which certainly fit the criterion.
On the otherhand, $\eta_|S| \sim 6$ for carbon, which one might be critical
about, and a correction is expected to be made dependent on the measurement
precision.

Especially the angular dependency of the scattered particles was investigated.
As projectiles scatter of both gold and carbon, different energies are
obtained. Accordingly, two gaussians; one for carbon and another for gold, was
fitted. For small angles the two gaussians merged which makes the seperation of
carbon and gold gaussians difficult, and thus the estimation of the centroid of
each more uncertain. On the other hand, for larger angles, the gaussians
seperate, why these are in better agreement with the theoretical expected
values.

For larger angles a small bump appears inbetween the carbon and gold
distributions. This is especially seen at $90$ and $100$ degrees. Accordingly,
the energies for these data has a bigger deviation from theory. For this, one
should not fit a double gaussian, but include a higher order; such as a tripple
gaussian fit. The carbon centroid would be shifted to higher channel numbers,
which correspondingly would increase the energy. 

Another interesting point is the energy changes of the scattered particles of
both carbon and gold, as a function of angle. The detected particles scattered
off of carbon has a rapid change of energy as a function of angle, whereas the
particles scattered off of gold remains almost independent of angle.
Considering the collision in the center of mass system, the reduced mass of
gold and the incomming particles is approximately the mass of a proton, as gold
is much heavier. Thus the kinetic energy of the proton is almost conserved, as
the energy absorbed in the recoil of gold is negligible. 

The layer thickness of a target were determined by changing which layer of the target that faces the incoming beam. The thickness of gold and carbon was found to be $250\pm25\si{\AA}$ and $1510\pm30\si{\AA}$ respectively. 

\vfill

