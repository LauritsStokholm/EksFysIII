\section{Conclusion}
During the experiment the Rutherford scattering has been examined. A relation between the differential cross section have been determined and compared with the theory assuming that the Summerfeld criterion is fulfilled. 


The classical Rutherford scattering of low energy hydrogenic ions, $H^+$ and $H_2^+$ on a double layered target of $\SI{20}{\angstrom}$ gold and $\SI{200}{\angstrom}$ carbon have been examined. Especially the angular dependency of the scattered particles was investigated. 


There is a clear relation between the fit of the data counts and the deviation of the theoretical expected energies.  

Accordingly, two gaussians; one for carbon and another for gold, was fitted. For small angles the two gaussians merged, making the estimation of the centroid more difficult. For larger angles, this problem is avoided. Another problem occur. We see a small bump in between the carbon and gold gaussians. Thus our assumption of a double gaussian is bad. This could be dodged had we fitted a tripple gaussian, which might shift the the carbon centroid towards higher bins and thus a higher energy. 



We assumed a linear combination of the two gaussian distributions. Nonetheless, this might have lead to 

