\section{Introduction} 
% <!-- TOOO: laurits READ AND WRITE Mon 03 Sep 2018 01:44:38 PM CEST -->

Almost all of our knowledge in the field of nuclear and atomic physics, has been
discovered by scattering experiments. Scattering theory underpins one of the
most ubiquitous tools in physics. 

This paper has a limited extend, and to keep our discussion simple and
relevant, we will only examine elastic collisions in the semi-classical regime,
governed by the Sommerfeld criterion for classical scattering.

This is usually fine for low energy physics, in which internal energies remain
constant and no further particles are created or annihilated. 

Our experiment involves a single Van-de-Graaf accellerator and the energy is in
the order of $\SI{400}{\kilo\electronvolt}$.

In low energy physics, scattering phenomena provide the standard tool to
explore solid state systems, and historically this was used as a first step
towards our current understanding of the atom.

