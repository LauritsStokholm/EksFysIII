\section{Introduction} 
In 1923, American physicist Arthur Holly Compton demonstrated the particle–like
nature of electromagnetic radiation by the effect now known as the Compton
effect. Compton conducted experiments in which high energy photons, such as
x–rays and gamma–rays, were scattered by atomic electrons. When the incoming
photon gives parts of its energy to the electron, then the scattered photon has
lower energy and according to The Planck Hypothesis, lower frequensies and
longer wavelength. The Compton effect is significant as it supported the
photoelectric effect in the early twenties, and demonstrated that light cannot
be explained purely as a wave phenomenon but rather as a duality of both wave
and particle.  This experiment serves to illustrate the Compton effect. 
