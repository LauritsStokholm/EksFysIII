\section{Introduction} 
Almost all of our knowledge in the field of nuclear and atomic physics has been discovered through scattering experiments, and the theory of scattering underpins one of the most ubiquitous tools in physics.
In low energy physics, scattering phenomena provide the standard tool to
explore solid state systems. Historically, this was used as a first step
towards our current understanding of the atom.

This paper examines the Rutherford scattering of a beam of
$\SI{350}{\kilo\electronvolt}$ protons on a thin foil of
a two layer $\mathrm{Au}$/$\mathrm{C}$ target. To limit the extend of the paper, and to
keep our discussion simple and relevant, we will only examine elastic
collisions in the semi-classical regime, governed by the Sommerfeld criterion
for classical scattering \parencite[p. 14]{noteBB}.

This is usually fine for low energy physics, in which internal energies remain
constant and no further particles are created or annihilated.
For this experiment, which uses a single Van-de-Graaff accelerator to generate particles with energies of up to $\SI{400}{\kilo\electronvolt}$, this is a good approximation.
