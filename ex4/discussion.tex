%\section{Discussion}
%
%\subsection{The experimental setup}
%The Van de Graaf accelerator provide an enormous advantage over the
%Cockcroft-Walton accelerator as the terminal voltage on a Van de Graaf is
%extreemely stable and does not ripple as the later does. This is very important
%when desired to measure reaction cross sections. Nonetheless, the Van de Graaf
%accelerator has a low current output in the $\si{\micro\ampere}$ range compared
%with the Cockcroft-Walton $\si{\milli\ampere}$. Nevertheless, this is quite
%sufficient for nuclear reaction experiments, and thus our chosen accelerator is
%the workhorse of low-energy nuclear structure physics.
%
%To improve the potential drop across the accelerator:
%
%Vacuum, so the limit of electrical breakdown (sparking) of 
%
%To reduce breakdown and sparking, the generator is enclosed in a pressure tank
%containing an insulating gas.
%Capacitance (geometrical size)
%\cite{krane}
%
%\subsection{Rutherford experiment}
%
%Provide data that test the angular dependence of the Rutherford cross section for scattering of 400kV protons on gold and carbon
%Measure the energy of the scattered protons on gold and carbon as a function of scattering angle 
%Provide data that test the Ztarget dependence of the Rutherford cross section
%Measure the thickness of the gold and carbon layers in the target.
%Challenge tasks :
%Provide data that checks the energy dependence of the Rutherford cross section
%Demonstrate that nuclear reactions occur when protons are incident on boron
%The measurement plan should as a minimum include 
%
%Calibration of the detector by using a pulser and one measured energy from the accelerator, or by using H+ and H2+ beams (do not direct the beam directly into the detector!).
%A plan to achieve the required tasks 1-4.
%
%Target dependency of the Rutherford cross section
%Nuclear reactions of protons with boron.

