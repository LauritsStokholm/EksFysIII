\section{Discussion}

\subsection{The experimental setup}
The Van de Graaf accelerator provide an enormous advantage over the
Cockcroft-Walton accelerator as the terminal voltage on a Van de Graaf is
extreemely stable and does not ripple as the later does. This is very important
when desired to measure reaction cross sections. Nonetheless, the Van de Graaf
accelerator has a low current output in the $\si{\micro\ampere}$ range compared
with the Cockcroft-Walton $\si{\milli\ampere}$. Nevertheless, this is quite
sufficient for nuclear reaction experiments, and thus our chosen accelerator is
the workhorse of low-energy nuclear structure physics.

To improve the potential drop across the accelerator:

Vacuum, so the limit of electrical breakdown (sparking) of 

To reduce breakdown and sparking, the generator is enclosed in a pressure tank
containing an insulating gas.
Capacitance (geometrical size)
\cite{krane}

\subsection{Energy}


