\section{Experimental Setup}
The following equipment, as described in this section, can be seen on
\fxnote{fig: setup}

\paragraph{BGO} In the experiment, there will be two detectors, the first of
which is called the BGO1–scintillator.  This is where the Compton scattering is
occuring. When the photon arrives at the BGO–scintillator, the BGO–crystal
absorbs the photon and therefore gets excited. After a short period of time,
the crystal will emit a new photon, essentially releasing almost all the energy
absorbed, at an angle.  After the Compton Scattering, the photon is absorbed by
the second detector, which leads us to our next item of equipment.

\paragraph{NaI} The second detector is the NaI2–scintillator. Whereas the BGO
almost emits all energy absorbed, the NaI absorbs the rest of the photon
energy. Otherwise they are principly described by the same mechanics.
Nonetheless, it is this important difference between the two detectors, that
makes the photon detectable as a readable signal, but to do so, the energy has
to be amplified.

\paragraph{Photo Amplifier}%
\label{par:photo_amplifier}
The photo amplifier is a necessity to read the photon energy as a pulse.
High–Voltage power suply To conduct the experiment, the high voltage power
suply is a necesity as well.
