This paper studies the Rutherford scattering of protons on atomic
nuclei. Hydrogenic ions of $\SI{350}{\kilo\electronvolt}$ were generated using
a Van de Graaff accelerator and directed onto thin metal foils of Au/C. The
scattered particles of different energies were detected at variable angles,
and the Rutherford differential cross section was meassured as a function of
the scattering angle in the range of $40$ to $160$ degrees. \\
The differential cross section showed a clear angular dependency and was in
general agreement with theory. The energy of the scattered particles was more consistent with the theory at higher angles. For lower angles the assumed model for fitting data was
imprecise, causing the mean energy of the scattered particles to be either lower or higher than expected. \\
The layer thickness of a target were determined by changing which layer of the target that faces the incoming beam. The thickness of gold and carbon was found to be $262\pm26\si{\AA}$ and $2325\pm51\si{\AA}$ respectively. 


