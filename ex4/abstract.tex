This paper studies the Rutherford scattering of protons on atomic
nuclei. Hydrogenic ions of $\SI{350}{\kilo\electronvolt}$ were generated using
a Van de Graaff accelerator and directed onto thin metal foils of Au/C. The
scattered particles of different energies were detected at variable angles,
and the Rutherford differential cross section was meassured as a function of
the scattering angle in the range of $40$ to $160$ degrees. 

The differential cross section showed a clear angular dependency and was in
general agreement with theory. At certain angles our assumed model was
inprecise, such the fit had a lower energy than expected. Twere meassured at
angles of $90$ and $100$ degrees. This is due to an inprecise as ....., as expected.
The thickness of the target layers Au/C were determined from the stopping power
of the layers to be ..... The nuclear reactions of protons with boron were
demonstrated by ... 
Mere is den dur bla bla bla ... 
In conclusion ... 


