This report studies the Rutherford scattering of protons on atomic
nuclei. For this, hydrogenic ions of $\SI{350}{\kilo\electronvolt}$ were
generated using
a Van de Graaff accelerator and directed onto thin metal foils of Au/C. The
scattered particles of different energies were detected at variable angles,
and the Rutherford differential cross section was measured as a function of
the scattering angle in the range of $40$ to $160$ degrees.
The differential cross section showed a clear angular dependency and was in
general agreement with theory. The energy of the scattered particles was more
consistent with the theory at higher angles. For lower angles the assumed model
for fitting data was
imprecise, causing the mean energy of the scattered particles to deviate from
the theoretical expected values.
The layer thickness of a target was determined by varying which layer of the
target that faces the incoming beam. The thickness of gold and carbon was found
to be $262\pm26\si{\angstrom}$ and $2325\pm51\si{\angstrom}$ respectively.

\vspace{1cm}
\begin{center}
\textbf{Division of labour}
\end{center}
\noindent
The report was written in full collaboration. Tasks were divided into
fragments and handled individually but always discussed and collectively
edited. This solution is in our group philosophy considered as the most
efficient and with result of a better and more complete rapport. Hence, all authors are responsible for all sections.




